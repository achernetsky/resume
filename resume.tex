\documentclass[11pt,a4paper]{moderncv}
\moderncvtheme[blue]{casual}
\usepackage[utf8]{inputenc}
\usepackage[pdftex]{hyperref}
\usepackage[scale=0.8]{geometry}
\usepackage{color}

\firstname{Alexander}
\familyname{Chernetsky}
\address{4, Ihnatoŭskaha str. Apt 45}{Minsk, Belarus 220055}
\mobile{+375~(29)~3683767}
\email{alexander.chernetsky@gmail.com}
\photo[64pt]{picture}

\begin{document}

\maketitle

\section{Education}
\cventry{2009-2014}{Mathematician–software degree}{BSUIR}{Minsk}{}{Graduated from \emph{Belarusian State University of Informatics and Radioelectronics} in \emph{2014}. Received \emph{mathematician--software} degree.}

\section{Diploma project}
\cvline{title}{\emph{Service to provide distributed locks and storage of configurations}}
\cvline{description}{Development of an service for obtaining distributed locks using \emph{RAFT} consensus algorithm was the goal of the diploma project. My work was based on the article ``In Search of an Understandable Consensus Algorithm'' by Diego Ongaro and John Ousterhout presented in Stanford University. I received a high grade for my diploma project and was offered to continue education to receive second stage of higher education, as well as teach students at ``Computer science'' department.}

\section{Experience}

\subsection{Gurtsoft (\url{http://gurtam.com}), Minsk}
\cventry{August 2012 - Present}{Developer of Gurtam GIS services}{}{}{}{Support of legacy code of GIS service such as Gurtam Maps, geocoding and reverse geocoding, also bug-fixing and implementation of new features.}
\cvline{tools}{\emph{Linux, C++, gcc, GNU make, Berkeley DB, git.}}

\cventry{}{Key developer of Gurtam routing service}{}{}{}{Development of service for computing optimal routes in road networks. I have developed so far the following key parts of system:
\begin{itemize}
  \item a construction of road graph from different sources (Polish Map Format, Shapefile format and OSM) and and merging them into a single unit;
  \item a preprocessing of Road Networks with Contraction Hierarchies;
  \item a shortest path computation in road networks with turn costs;
\end{itemize}}
\cvline{tools}{\emph{Linux, C++, Boost, OMP, gcc, GNU make, Berkeley DB, git.}}

\cventry{}{Key developer of Wialon apps}{}{}{}{I have implemented a few applications. Among them are Driving Logbook, Dashboard, Gurtam Maps and iDriveSafe IQ. They are implemented using Wialon Hosting and Wialon Kit SDKs. Driving Logbook allows requesting reports/a report for a unit's trips with an opportunity of adjusting trip status and making user's notes. Dashboard provides overview of fleet KPIs (key performance indicators) in an easy to read real-time user interface to enable instantaneous and informed decisions to be made at a glance. Gurtam Maps can be used in two ways: to search location and address by given coordinates and to search location and coordinates by given address or its part. iDriveSafe IQ is a comprehensive driving quality control system that allows to evaluate drivers’ behavior and the way they treat their vehicles.}
\cvline{tools}{\emph{Javascript, Backbone, RequireJS, git.}}

\cventry{}{Key developer of \href{http://goo.gl/62FHRn}{GPS Tag Pro} and \href{http://goo.gl/sMMB3W}{GPS Tag Orange} iOS apps}{}{}{}{Application that helps determine your geographical position and track your movements. Full application development, bug-fixing, support of legacy code.}
\cvline{tools}{\emph{Xcode, Objective C, SQLite, git}}

\cventry{}{Developer of hosting.wialon.com}{}{}{}{Web-based GPS tracking software platform. Implementation of new features, bug-fixing, support of legacy code.}
\cvline{tools}{\emph{Linux, JavaScript, Qooxdoo, Tcl, git.}}

\cventry{}{Key developer infrastructure for gps.velcom.by}{}{}{}{Personal GPS tracking web-based service. Development of the website and provisioning API, bug-fixing, support of legacy code. At the same time I implemeted simple Python wrapper for Wialon Remote API.}
\cvline{tools}{\emph{Linux, JavaScript, Qooxdoo, Python, Tornado, Fabric, Tcl, git.}}

\subsection{Y-NODE Software (\url{http://y-node.com}), Minsk}

\cventry{June 2011 -\\August 2012}{Developer of \href{http://oekostrom-fuer-alle.de/}{oekostrom-fuer-alle.de}}{}{}{}{Official site for \emph{ÖfA Ökostrom für Alle GmbH company}. Implementation of new features, bug-fixing, support of legacy code.}
\cvline{tools}{\emph{Linux, Python, Django, jQuery, PostgreSQL, git, memchached.}}

\cventry{}{Key developer of \href{http://goo.gl/QRLcG6}{Ökostrom WechselApp}}{}{}{}{Official iOS app for \emph{ÖfA Ökostrom für Alle GmbH} company. Full development of the app, and server-side programming using Django.}
\cvline{tools}{\emph{Objective C, Xcode, Python, Django, Javascript, PostgreSQL.}}

\cventry{}{Developer of \href{http://invitebox.com}{invitebox.com}}{}{}{}{InviteBox is a widget for swift referral promotions over social networks, email, and mobile phones. Implementation of new features, bug-fixing, support of legacy code.}
\cvline{tools}{\emph{Linux, Python, Django, jQuery, PostgreSQL, DjangoCMS.}}

\cventry{}{Developer of \href{http://notsharingmy.info}{notsharingmy.info}}{}{}{}{Web anonymizer which --- among other features -- generates anonymous emails for users so that they do not disclose their personal ones. Implementation of new features and protection mechanisms, bug-fixing.}
\cvline{tools}{\emph{Linux, Django, Python, MongoDB, jQuery.}}

\section{On the Web}
\cvlistitem{\url{http://github.com/achernetsky}}
\cvlistitem{\url{http://linkedin.com/in/achernetsky}}

\section{Languages}
\cvline{\textbf{English}}{Pre-Intermediate}
\cvline{\textbf{Russian}}{Native}

\section{Tools used at work}
\cvcomputer{\textbf{Languages}}{Objective C, C++, Python, JavaScript}{\textbf{Operating systems}}{Linux}
\cvcomputer{\textbf{Databases}}{PostgreSQL, SQLite, MongoDB, Berkeley DB}{\textbf{Mobile development}}{iOS}
\cvline{\textbf{Revision control}}{git, svn}

\section{Tools used in personal/educational projects}
\cvcomputer{\textbf{Languages}}{C, Java, Scala, Scheme,\\ Octave, assembly language}{\textbf{Databases}}{MySQL, CouchDB}

\section{Interests}
\cvlistdoubleitem{Functional programming}{Mobile development}
\cvlistdoubleitem{Web development}{}

\end{document}
